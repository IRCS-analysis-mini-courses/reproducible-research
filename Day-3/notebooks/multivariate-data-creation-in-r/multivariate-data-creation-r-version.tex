\documentclass[]{article}
\usepackage{lmodern}
\usepackage{amssymb,amsmath}
\usepackage{ifxetex,ifluatex}
\usepackage{fixltx2e} % provides \textsubscript
\ifnum 0\ifxetex 1\fi\ifluatex 1\fi=0 % if pdftex
  \usepackage[T1]{fontenc}
  \usepackage[utf8]{inputenc}
\else % if luatex or xelatex
  \ifxetex
    \usepackage{mathspec}
    \usepackage{xltxtra,xunicode}
  \else
    \usepackage{fontspec}
  \fi
  \defaultfontfeatures{Mapping=tex-text,Scale=MatchLowercase}
  \newcommand{\euro}{€}
\fi
% use upquote if available, for straight quotes in verbatim environments
\IfFileExists{upquote.sty}{\usepackage{upquote}}{}
% use microtype if available
\IfFileExists{microtype.sty}{%
\usepackage{microtype}
\UseMicrotypeSet[protrusion]{basicmath} % disable protrusion for tt fonts
}{}
\usepackage[margin=1in]{geometry}
\usepackage{color}
\usepackage{fancyvrb}
\newcommand{\VerbBar}{|}
\newcommand{\VERB}{\Verb[commandchars=\\\{\}]}
\DefineVerbatimEnvironment{Highlighting}{Verbatim}{commandchars=\\\{\}}
% Add ',fontsize=\small' for more characters per line
\usepackage{framed}
\definecolor{shadecolor}{RGB}{248,248,248}
\newenvironment{Shaded}{\begin{snugshade}}{\end{snugshade}}
\newcommand{\KeywordTok}[1]{\textcolor[rgb]{0.13,0.29,0.53}{\textbf{{#1}}}}
\newcommand{\DataTypeTok}[1]{\textcolor[rgb]{0.13,0.29,0.53}{{#1}}}
\newcommand{\DecValTok}[1]{\textcolor[rgb]{0.00,0.00,0.81}{{#1}}}
\newcommand{\BaseNTok}[1]{\textcolor[rgb]{0.00,0.00,0.81}{{#1}}}
\newcommand{\FloatTok}[1]{\textcolor[rgb]{0.00,0.00,0.81}{{#1}}}
\newcommand{\CharTok}[1]{\textcolor[rgb]{0.31,0.60,0.02}{{#1}}}
\newcommand{\StringTok}[1]{\textcolor[rgb]{0.31,0.60,0.02}{{#1}}}
\newcommand{\CommentTok}[1]{\textcolor[rgb]{0.56,0.35,0.01}{\textit{{#1}}}}
\newcommand{\OtherTok}[1]{\textcolor[rgb]{0.56,0.35,0.01}{{#1}}}
\newcommand{\AlertTok}[1]{\textcolor[rgb]{0.94,0.16,0.16}{{#1}}}
\newcommand{\FunctionTok}[1]{\textcolor[rgb]{0.00,0.00,0.00}{{#1}}}
\newcommand{\RegionMarkerTok}[1]{{#1}}
\newcommand{\ErrorTok}[1]{\textbf{{#1}}}
\newcommand{\NormalTok}[1]{{#1}}
\usepackage{graphicx}
\makeatletter
\def\maxwidth{\ifdim\Gin@nat@width>\linewidth\linewidth\else\Gin@nat@width\fi}
\def\maxheight{\ifdim\Gin@nat@height>\textheight\textheight\else\Gin@nat@height\fi}
\makeatother
% Scale images if necessary, so that they will not overflow the page
% margins by default, and it is still possible to overwrite the defaults
% using explicit options in \includegraphics[width, height, ...]{}
\setkeys{Gin}{width=\maxwidth,height=\maxheight,keepaspectratio}
\ifxetex
  \usepackage[setpagesize=false, % page size defined by xetex
              unicode=false, % unicode breaks when used with xetex
              xetex]{hyperref}
\else
  \usepackage[unicode=true]{hyperref}
\fi
\hypersetup{breaklinks=true,
            bookmarks=true,
            pdfauthor={Julian},
            pdftitle={Creating the large multivariate dataset in R},
            colorlinks=true,
            citecolor=blue,
            urlcolor=blue,
            linkcolor=magenta,
            pdfborder={0 0 0}}
\urlstyle{same}  % don't use monospace font for urls
\setlength{\parindent}{0pt}
\setlength{\parskip}{6pt plus 2pt minus 1pt}
\setlength{\emergencystretch}{3em}  % prevent overfull lines
\setcounter{secnumdepth}{5}

%%% Use protect on footnotes to avoid problems with footnotes in titles
\let\rmarkdownfootnote\footnote%
\def\footnote{\protect\rmarkdownfootnote}

%%% Change title format to be more compact
\usepackage{titling}

% Create subtitle command for use in maketitle
\newcommand{\subtitle}[1]{
  \posttitle{
    \begin{center}\large#1\end{center}
    }
}

\setlength{\droptitle}{-2em}
  \title{Creating the large multivariate dataset in R}
  \pretitle{\vspace{\droptitle}\centering\huge}
  \posttitle{\par}
  \author{Julian}
  \preauthor{\centering\large\emph}
  \postauthor{\par}
  \date{}
  \predate{}\postdate{}



\begin{document}

\maketitle


{
\hypersetup{linkcolor=black}
\setcounter{tocdepth}{2}
\tableofcontents
}
\section{Executive summary}\label{executive-summary}

The goal of this document:\\- demonstrate various R commands\\- how to
potentially import and inspect a large dataset\\- creating statistical
models for large datasets\\- how to author a report using Sweave or R
Markdown

\section{Steps}\label{steps}

\begin{itemize}
\itemsep1pt\parskip0pt\parsep0pt
\item
  create subject ids\\
\item
  set a random seed (makes tracking what was done harder)\\
\item
  randomly creating subject ages\\
\item
  simulating subject responses\\
\item
  generating reaction times and introducing real data\\
\item
  assigning attempts\\
\item
  assigning subjects to groups\\
\item
  adding trial, token phase data in long format
\end{itemize}

\section{Execution}\label{execution}

This document was created in R Markdown. All the data were generated
using R code. To execute the commands, change the chunk options to
\texttt{eval\ \ =\ TRUE}.

The source code (R) is contained in the Day 3 directory.

\includegraphics{ozymandias-by-gibbons.jpg}

\section{Data generation}\label{data-generation}

\subsection{Set random seed}\label{set-random-seed}

\begin{Shaded}
\begin{Highlighting}[]
\KeywordTok{RNGkind}\NormalTok{(}\StringTok{'Mersenne-Twister'}\NormalTok{)}
\KeywordTok{set.seed}\NormalTok{(}\DecValTok{2340875}\NormalTok{)}
\NormalTok{old.seed <-}\StringTok{ }\NormalTok{.Random.seed}
\end{Highlighting}
\end{Shaded}

\subsection{\texorpdfstring{Create \texttt{Subject}, \texttt{Age},
\texttt{Group}
vectors}{Create Subject, Age, Group vectors}}\label{create-subject-age-group-vectors}

\begin{Shaded}
\begin{Highlighting}[]
\NormalTok{subj.nums <-}\StringTok{ }\KeywordTok{seq}\NormalTok{(}\DecValTok{1}\NormalTok{, }\DecValTok{102}\NormalTok{, }\DecValTok{1}\NormalTok{)}

\NormalTok{subj.ids <-}\StringTok{ }\KeywordTok{character}\NormalTok{(}\KeywordTok{length}\NormalTok{(subj.nums))}

\NormalTok{for (i in }\DecValTok{1}\NormalTok{:}\KeywordTok{length}\NormalTok{(subj.nums)) \{}
  \NormalTok{subj.ids[i] <-}\StringTok{ }\KeywordTok{paste}\NormalTok{(}\StringTok{'subj-'}\NormalTok{, subj.nums[i], }\DataTypeTok{sep =} \StringTok{''}\NormalTok{)}
\NormalTok{\}}

\NormalTok{subj.rep <-}\StringTok{ }\KeywordTok{rep}\NormalTok{(subj.ids, }\DecValTok{800}\NormalTok{)}

\NormalTok{groups <-}\StringTok{ }\KeywordTok{list}\NormalTok{(}\StringTok{'treatment-1'}\NormalTok{, }\StringTok{'treatment-2'}\NormalTok{, }\StringTok{'control'}\NormalTok{)}
\NormalTok{age.vec <-}\StringTok{ }\KeywordTok{runif}\NormalTok{(}\DataTypeTok{min =} \DecValTok{18}\NormalTok{, }\DataTypeTok{max =} \DecValTok{52}\NormalTok{, }\DataTypeTok{n =} \KeywordTok{length}\NormalTok{(subj.ids))}
\end{Highlighting}
\end{Shaded}

\subsection{Assign subjects to an age}\label{assign-subjects-to-an-age}

Replicates the ages to match the subject repetitions

\begin{Shaded}
\begin{Highlighting}[]
\NormalTok{age.rep <-}\StringTok{ }\KeywordTok{rep}\NormalTok{(age.vec, }\DecValTok{800}\NormalTok{)}
\end{Highlighting}
\end{Shaded}

\subsection{Simulating correct
responses}\label{simulating-correct-responses}

Much slower than generating vector (especially since R is making a deep
copy each time), but harder to figure out what was done

\begin{Shaded}
\begin{Highlighting}[]
\NormalTok{correctness <-}\StringTok{ }\KeywordTok{runif}\NormalTok{(}\DataTypeTok{min =} \DecValTok{0}\NormalTok{, }\DataTypeTok{max  =} \DecValTok{1}\NormalTok{, }\DataTypeTok{n =} \KeywordTok{length}\NormalTok{(subj.rep))}

\NormalTok{correct.vec <-}\StringTok{ }\KeywordTok{integer}\NormalTok{()}

\NormalTok{for (i in }\DecValTok{1}\NormalTok{:}\KeywordTok{length}\NormalTok{(subj.rep)) \{}
  \NormalTok{correct.vec[i] <-}\StringTok{ }\KeywordTok{rbinom}\NormalTok{(}\DataTypeTok{n =} \DecValTok{1}\NormalTok{, }\DataTypeTok{prob =} \NormalTok{correctness[i], }\DataTypeTok{size =} \DecValTok{1}\NormalTok{)}
\NormalTok{\}}
\end{Highlighting}
\end{Shaded}

\subsection{Generating reaction times}\label{generating-reaction-times}

Once again, very slow but an extra hurdle to jump through

\subsection{Replacing values selectively with real
data}\label{replacing-values-selectively-with-real-data}

\begin{Shaded}
\begin{Highlighting}[]
\NormalTok{data.dir <-}\StringTok{ '~/GitHub/reproducible-research/Day-2/datasets'}
\NormalTok{data.file <-}\StringTok{ 'psycho-data-april-2015.csv'}

\NormalTok{real.data <-}\StringTok{ }\KeywordTok{read.csv}\NormalTok{(}\KeywordTok{file.path}\NormalTok{(data.dir, data.file), }\DataTypeTok{header =} \OtherTok{TRUE}\NormalTok{)}

\NormalTok{real.rt <-}\StringTok{ }\NormalTok{real.data$absRT}

\NormalTok{for (i in }\DecValTok{1}\NormalTok{:}\KeywordTok{length}\NormalTok{(subj.rep)) \{}
  \NormalTok{if (rt.vec[i] >}\StringTok{ }\DecValTok{600}\NormalTok{) \{}
    \NormalTok{rt.vec[i] <-}\StringTok{ }\KeywordTok{sample}\NormalTok{(real.rt, }\DataTypeTok{replace =} \OtherTok{TRUE}\NormalTok{, }\DataTypeTok{size =} \DecValTok{1}\NormalTok{)}
  \NormalTok{\} else \{}
    \NormalTok{next }\CommentTok{# R's version of a continue statement}
  \NormalTok{\}}
\NormalTok{\}}
\end{Highlighting}
\end{Shaded}

\subsection{Creating timed out values}\label{creating-timed-out-values}

\begin{Shaded}
\begin{Highlighting}[]
\NormalTok{for (i in }\DecValTok{1}\NormalTok{:}\KeywordTok{length}\NormalTok{(subj.rep)) \{}
  \NormalTok{if (rt.vec[i] >}\StringTok{ }\DecValTok{2200}\NormalTok{) \{}
    \NormalTok{rt.vec[i] =}\StringTok{ }\DecValTok{200}
  \NormalTok{\} else \{}
    \NormalTok{next}
  \NormalTok{\}}
\NormalTok{\}}
\end{Highlighting}
\end{Shaded}

\#\# Creating attempts vector Since there is no random integer function
in base R, round numbers from a uniform distribution

\begin{Shaded}
\begin{Highlighting}[]
\NormalTok{attempt.vec <-}\StringTok{ }\KeywordTok{round}\NormalTok{(}\KeywordTok{runif}\NormalTok{(}\DataTypeTok{n =} \KeywordTok{length}\NormalTok{(subj.rep), }\DataTypeTok{min =} \DecValTok{1}\NormalTok{, }\DataTypeTok{max =} \DecValTok{15}\NormalTok{))}
\end{Highlighting}
\end{Shaded}

\subsection{Create token vector}\label{create-token-vector}

\begin{Shaded}
\begin{Highlighting}[]
\NormalTok{token.rep <-}\StringTok{ }\KeywordTok{rep}\NormalTok{(}\KeywordTok{seq}\NormalTok{(}\DecValTok{1}\NormalTok{, }\DecValTok{20}\NormalTok{, }\DecValTok{1}\NormalTok{), }\DataTypeTok{each =} \DecValTok{20}\NormalTok{)}
\NormalTok{token.vec <-}\StringTok{ }\KeywordTok{rep}\NormalTok{(token.rep, }\DecValTok{800}\NormalTok{)}
\end{Highlighting}
\end{Shaded}

\subsection{\texorpdfstring{Create experimental
\texttt{data.frame}}{Create experimental data.frame}}\label{create-experimental-data.frame}

Using \texttt{data.table} since it is faster

\begin{Shaded}
\begin{Highlighting}[]
\KeywordTok{library}\NormalTok{(data.table)}

\NormalTok{sample.data$Subject <-}\StringTok{ }\NormalTok{subj.rep}

\NormalTok{sample.data <-}\StringTok{ }
\StringTok{  }\KeywordTok{data.table}\NormalTok{(}\DataTypeTok{Subject =} \NormalTok{subj.rep, }\DataTypeTok{Age =} \NormalTok{age.rep, }\DataTypeTok{Attempts =} \NormalTok{attempt.vec, }
             \DataTypeTok{Correct =} \NormalTok{correct.vec, }\DataTypeTok{RT =} \NormalTok{rt.vec)}
\end{Highlighting}
\end{Shaded}

\subsection{Add in trials}\label{add-in-trials}

For the trials, tokens, groups, and phases a new function will be
introduced: \texttt{gl()}. This allows you to generate factor levels and
it will be used to create the labels for each level. Generally, you
would use this to create new factors with appropriate levels for
statistical analysis, but this is a good way to introduce this
function.\\It should be noted that you should be careful with this
function, as you will need to know the pattern for the labels you are
creating.

The first step will be to sort the data into long format (well, the
convention anyway). This makes it easier to see what is going on with
the levels. You shold also take advantage of the viewer (\texttt{View()}
function) in RStudio to explore the data structure.

\subsection{Sorting data into proper long
format}\label{sorting-data-into-proper-long-format}

Easiest to sort the data using \texttt{dplyr} (though \texttt{plyr}
would also work well).

\begin{Shaded}
\begin{Highlighting}[]
\KeywordTok{library}\NormalTok{(dplyr)}
\NormalTok{sample.data <-}\StringTok{ }\KeywordTok{arrange}\NormalTok{(sample.data, Subject)}
\end{Highlighting}
\end{Shaded}

\subsection{Generate tokens}\label{generate-tokens}

\begin{Shaded}
\begin{Highlighting}[]
\NormalTok{sample.data$Token <-}\StringTok{ }\KeywordTok{gl}\NormalTok{(}\DataTypeTok{n =} \DecValTok{20}\NormalTok{, }\DataTypeTok{k =} \DecValTok{40}\NormalTok{, }\DataTypeTok{labels =} \KeywordTok{seq}\NormalTok{(}\DecValTok{1}\NormalTok{, }\DecValTok{20}\NormalTok{, }\DecValTok{1}\NormalTok{))}
\end{Highlighting}
\end{Shaded}

\subsection{Generate trials}\label{generate-trials}

\begin{Shaded}
\begin{Highlighting}[]
\NormalTok{sample.data$Trial <-}\StringTok{ }\KeywordTok{gl}\NormalTok{(}\DataTypeTok{n =} \DecValTok{20}\NormalTok{, }\DataTypeTok{k =} \DecValTok{1}\NormalTok{, }\DataTypeTok{length =} \DecValTok{20} \NormalTok{*}\StringTok{ }\KeywordTok{length}\NormalTok{(subj.ids))}
\end{Highlighting}
\end{Shaded}

\subsection{Generate phases}\label{generate-phases}

\begin{Shaded}
\begin{Highlighting}[]
\NormalTok{sample.data$Phase <-}\StringTok{ }\KeywordTok{gl}\NormalTok{(}\DataTypeTok{n =} \DecValTok{2}\NormalTok{, }\DataTypeTok{k =} \DecValTok{400}\NormalTok{, }\DataTypeTok{labels =} \KeywordTok{c}\NormalTok{(}\DecValTok{1}\NormalTok{, }\DecValTok{2}\NormalTok{))}
\end{Highlighting}
\end{Shaded}

\subsection{Generate groups}\label{generate-groups}

The groups will not be generated in the same manner as was done in
Python. I could have mapped values out previously, but instead I will
just assign

\subsection{Changing data to appropriate
types}\label{changing-data-to-appropriate-types}

\subsection{Export to CSV}\label{export-to-csv}

\section{Appendix: vectorized
functions}\label{appendix-vectorized-functions}

The data were created in a rather slow, tedious manner, mostly to
introduce control flow and to be explicit about the steps that were
taken. This is further complicated by the fact that R handles data in a
less than ideal way sometimes, especially with making deep copies on
loops \href{http://adv-r.had.co.nz/memory.html}{see here for more
details}.

Below are examples of using vectorized functions instead of \texttt{for}
loops to manipulate data. There are also additional packages (such as
\texttt{data.table}) that vectorize functions and work with large
datasets.

\subsection{Simulating correctness}\label{simulating-correctness}

Here we use the vector of probabilities and specify the size of the
final vector to speed up computation:

\begin{Shaded}
\begin{Highlighting}[]
\NormalTok{correct.vec <-}\StringTok{ }\KeywordTok{rbinom}\NormalTok{(}\DataTypeTok{n =} \DecValTok{81600}\NormalTok{, }\DataTypeTok{prob =} \NormalTok{correctness, }\DataTypeTok{size =} \DecValTok{1}\NormalTok{)}
\end{Highlighting}
\end{Shaded}

\subsection{Simulating reaction time}\label{simulating-reaction-time}

Generating the reaction times is done in the same manner:

\begin{Shaded}
\begin{Highlighting}[]
\NormalTok{rt.vec <-}\StringTok{ }
\StringTok{  }\KeywordTok{rlnorm}\NormalTok{(}\DataTypeTok{n =} \DecValTok{81600}\NormalTok{, }\DataTypeTok{meanlog =} \FloatTok{6.0}\NormalTok{, }\DataTypeTok{sdlog =} \FloatTok{0.5}\NormalTok{) +}\StringTok{ }\DecValTok{5} \NormalTok{+}
\StringTok{    }\KeywordTok{runif}\NormalTok{(}\DataTypeTok{n =} \DecValTok{81600}\NormalTok{, }\DataTypeTok{min =} \DecValTok{10}\NormalTok{, }\DataTypeTok{max =} \DecValTok{100}\NormalTok{) +}
\StringTok{    }\KeywordTok{rnorm}\NormalTok{(}\DataTypeTok{n =} \DecValTok{81600}\NormalTok{, }\DataTypeTok{mean =} \DecValTok{70}\NormalTok{, }\DataTypeTok{sd =} \DecValTok{15}\NormalTok{)}
\end{Highlighting}
\end{Shaded}

\subsection{Replacing simulated values with real
values}\label{replacing-simulated-values-with-real-values}

There are several options here that we could go with. We will use base R
functions for indexing, replacement and sampling.

\begin{Shaded}
\begin{Highlighting}[]
\NormalTok{gr}\FloatTok{.600} \NormalTok{<-}\StringTok{ }\KeywordTok{which}\NormalTok{(rt.vec >}\StringTok{ }\DecValTok{600}\NormalTok{)}

\NormalTok{real.sample <-}\StringTok{ }\KeywordTok{sample}\NormalTok{(real.rt, }\DataTypeTok{replace =} \OtherTok{TRUE}\NormalTok{, }\DataTypeTok{size =} \KeywordTok{length}\NormalTok{(gr}\FloatTok{.600}\NormalTok{))}

\NormalTok{rt.vec <-}\StringTok{ }\KeywordTok{replace}\NormalTok{(rt.vec, gr}\FloatTok{.600}\NormalTok{, real.sample)}
\end{Highlighting}
\end{Shaded}

\subsection{Creating timed out
values}\label{creating-timed-out-values-1}

Follow the same procedure as the above to create the timed out values:

\begin{Shaded}
\begin{Highlighting}[]
\NormalTok{gr}\FloatTok{.2200} \NormalTok{<-}\StringTok{ }\KeywordTok{which}\NormalTok{(rt.vec >}\StringTok{ }\DecValTok{2200}\NormalTok{)}

\NormalTok{rt.vec <-}\StringTok{ }\KeywordTok{replace}\NormalTok{(rt.vec, gr}\FloatTok{.2200}\NormalTok{, }\DecValTok{2200}\NormalTok{)}
\end{Highlighting}
\end{Shaded}

\end{document}

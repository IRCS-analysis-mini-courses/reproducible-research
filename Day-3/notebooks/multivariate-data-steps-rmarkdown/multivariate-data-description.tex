\documentclass[]{article}
\usepackage{lmodern}
\usepackage{amssymb,amsmath}
\usepackage{ifxetex,ifluatex}
\usepackage{fixltx2e} % provides \textsubscript
\ifnum 0\ifxetex 1\fi\ifluatex 1\fi=0 % if pdftex
  \usepackage[T1]{fontenc}
  \usepackage[utf8]{inputenc}
\else % if luatex or xelatex
  \ifxetex
    \usepackage{mathspec}
    \usepackage{xltxtra,xunicode}
  \else
    \usepackage{fontspec}
  \fi
  \defaultfontfeatures{Mapping=tex-text,Scale=MatchLowercase}
  \newcommand{\euro}{€}
\fi
% use upquote if available, for straight quotes in verbatim environments
\IfFileExists{upquote.sty}{\usepackage{upquote}}{}
% use microtype if available
\IfFileExists{microtype.sty}{%
\usepackage{microtype}
\UseMicrotypeSet[protrusion]{basicmath} % disable protrusion for tt fonts
}{}
\usepackage[margin=1in]{geometry}
\usepackage{color}
\usepackage{fancyvrb}
\newcommand{\VerbBar}{|}
\newcommand{\VERB}{\Verb[commandchars=\\\{\}]}
\DefineVerbatimEnvironment{Highlighting}{Verbatim}{commandchars=\\\{\}}
% Add ',fontsize=\small' for more characters per line
\usepackage{framed}
\definecolor{shadecolor}{RGB}{248,248,248}
\newenvironment{Shaded}{\begin{snugshade}}{\end{snugshade}}
\newcommand{\KeywordTok}[1]{\textcolor[rgb]{0.13,0.29,0.53}{\textbf{{#1}}}}
\newcommand{\DataTypeTok}[1]{\textcolor[rgb]{0.13,0.29,0.53}{{#1}}}
\newcommand{\DecValTok}[1]{\textcolor[rgb]{0.00,0.00,0.81}{{#1}}}
\newcommand{\BaseNTok}[1]{\textcolor[rgb]{0.00,0.00,0.81}{{#1}}}
\newcommand{\FloatTok}[1]{\textcolor[rgb]{0.00,0.00,0.81}{{#1}}}
\newcommand{\CharTok}[1]{\textcolor[rgb]{0.31,0.60,0.02}{{#1}}}
\newcommand{\StringTok}[1]{\textcolor[rgb]{0.31,0.60,0.02}{{#1}}}
\newcommand{\CommentTok}[1]{\textcolor[rgb]{0.56,0.35,0.01}{\textit{{#1}}}}
\newcommand{\OtherTok}[1]{\textcolor[rgb]{0.56,0.35,0.01}{{#1}}}
\newcommand{\AlertTok}[1]{\textcolor[rgb]{0.94,0.16,0.16}{{#1}}}
\newcommand{\FunctionTok}[1]{\textcolor[rgb]{0.00,0.00,0.00}{{#1}}}
\newcommand{\RegionMarkerTok}[1]{{#1}}
\newcommand{\ErrorTok}[1]{\textbf{{#1}}}
\newcommand{\NormalTok}[1]{{#1}}
\usepackage{graphicx}
\makeatletter
\def\maxwidth{\ifdim\Gin@nat@width>\linewidth\linewidth\else\Gin@nat@width\fi}
\def\maxheight{\ifdim\Gin@nat@height>\textheight\textheight\else\Gin@nat@height\fi}
\makeatother
% Scale images if necessary, so that they will not overflow the page
% margins by default, and it is still possible to overwrite the defaults
% using explicit options in \includegraphics[width, height, ...]{}
\setkeys{Gin}{width=\maxwidth,height=\maxheight,keepaspectratio}
\ifxetex
  \usepackage[setpagesize=false, % page size defined by xetex
              unicode=false, % unicode breaks when used with xetex
              xetex]{hyperref}
\else
  \usepackage[unicode=true]{hyperref}
\fi
\hypersetup{breaklinks=true,
            bookmarks=true,
            pdfauthor={Julian},
            pdftitle={Creating the large multivariate dataset},
            colorlinks=true,
            citecolor=blue,
            urlcolor=blue,
            linkcolor=magenta,
            pdfborder={0 0 0}}
\urlstyle{same}  % don't use monospace font for urls
\setlength{\parindent}{0pt}
\setlength{\parskip}{6pt plus 2pt minus 1pt}
\setlength{\emergencystretch}{3em}  % prevent overfull lines
\setcounter{secnumdepth}{0}

%%% Use protect on footnotes to avoid problems with footnotes in titles
\let\rmarkdownfootnote\footnote%
\def\footnote{\protect\rmarkdownfootnote}

%%% Change title format to be more compact
\usepackage{titling}

% Create subtitle command for use in maketitle
\newcommand{\subtitle}[1]{
  \posttitle{
    \begin{center}\large#1\end{center}
    }
}

\setlength{\droptitle}{-2em}
  \title{Creating the large multivariate dataset}
  \pretitle{\vspace{\droptitle}\centering\huge}
  \posttitle{\par}
  \author{Julian}
  \preauthor{\centering\large\emph}
  \postauthor{\par}
  \date{}
  \predate{}\postdate{}



\begin{document}

\maketitle


{
\hypersetup{linkcolor=black}
\setcounter{tocdepth}{2}
\tableofcontents
}
\section{Executive summary}\label{executive-summary}

The goal of creating the multivariate dataset was to:\\- demonstrate
various Python commands\\- how to potentially import and inspect a large
dataset\\- creating statistical models for large datasets\\- how to
author a report using Sweave or R Markdown

\section{Steps}\label{steps}

\begin{itemize}
\itemsep1pt\parskip0pt\parsep0pt
\item
  create subject ids\\
\item
  set a random seed (makes tracking what was done harder)\\
\item
  randomly creating subject ages\\
\item
  simulating subject responses\\
\item
  generating reaction times and introducing real data\\
\item
  assigning attempts\\
\item
  assigning subjects to groups\\
\item
  adding trial, token phase data in long format
\end{itemize}

The source code (Python) is contained in the Day 3 directory.

\includegraphics{ozymandias-by-gibbons.jpg}

\section{Execution}\label{execution}

This document was created in R Markdown. When used in conjunction with
\texttt{knitr}, you can use other engines such as Python. So you could
take this file, change the chunk options to \texttt{eval\ \ =\ TRUE} and
replicate the results without having to do so in Python. It much easier
however, to examine the source file posted.

\section{Data generation}\label{data-generation}

\subsection{\texorpdfstring{Import \texttt{NumPy}, \texttt{pandas}, set
random
seed}{Import NumPy, pandas, set random seed}}\label{import-numpy-pandas-set-random-seed}

\begin{Shaded}
\begin{Highlighting}[]

\CharTok{import} \NormalTok{numpy }\CharTok{as} \NormalTok{np}
\CharTok{import} \NormalTok{pandas }\CharTok{as} \NormalTok{pd}
\NormalTok{np.random.seed(}\DecValTok{2340875}\NormalTok{)}
\end{Highlighting}
\end{Shaded}

\subsection{\texorpdfstring{Create \texttt{Subject}, \texttt{Age},
\texttt{Group}
vectors}{Create Subject, Age, Group vectors}}\label{create-subject-age-group-vectors}

\begin{Shaded}
\begin{Highlighting}[]

\NormalTok{subj_ids = [}\StringTok{'subj-'} \NormalTok{+ }\DataTypeTok{str}\NormalTok{(num) }\KeywordTok{for} \NormalTok{num in }\DataTypeTok{range}\NormalTok{(}\DecValTok{1}\NormalTok{, }\DecValTok{103}\NormalTok{)]}
\NormalTok{groups = [}\StringTok{'treatment-1'}\NormalTok{, }\StringTok{'treatment-2'}\NormalTok{, }\StringTok{'control'}\NormalTok{]}
\NormalTok{age_vec = np.random.uniform(low = }\DecValTok{18}\NormalTok{, high = }\DecValTok{52}\NormalTok{, size = }\DataTypeTok{len}\NormalTok{(subj_ids))}
\end{Highlighting}
\end{Shaded}

\subsection{\texorpdfstring{Create experimental
\texttt{DataFrame}}{Create experimental DataFrame}}\label{create-experimental-dataframe}

\begin{Shaded}
\begin{Highlighting}[]
\NormalTok{cols = [}\StringTok{'Subject'}\NormalTok{, }\StringTok{'Age'}\NormalTok{, }\StringTok{'Group'}\NormalTok{, }\StringTok{'Token'}\NormalTok{,}\StringTok{'Attempts'}\NormalTok{, }\StringTok{'Correct'}\NormalTok{, }\StringTok{'RT'}\NormalTok{]}
\NormalTok{sample_data = pd.DataFrame(columns = cols)}
\NormalTok{sample_data[}\StringTok{'Subject'}\NormalTok{] = subj_ids * }\DecValTok{20} \NormalTok{* }\DecValTok{40}
\NormalTok{sample_data.head()}
\end{Highlighting}
\end{Shaded}

\subsection{Assign subjects to an age}\label{assign-subjects-to-an-age}

\begin{Shaded}
\begin{Highlighting}[]
\NormalTok{age_dict = \{\}}
\KeywordTok{for} \NormalTok{subj, age in }\DataTypeTok{zip}\NormalTok{(subj_ids, age_vec):}
    \NormalTok{age_dict[subj] = age}
\end{Highlighting}
\end{Shaded}

\subsection{Simulating correct
responses}\label{simulating-correct-responses}

Much slower than generating vector, but harder to figure out what was
done

\begin{Shaded}
\begin{Highlighting}[]
\NormalTok{correctness = np.random.uniform(low = }\DecValTok{0}\NormalTok{, high = }\DecValTok{1}\NormalTok{, size = sample_data.shape[}\DecValTok{0}\NormalTok{])}

\KeywordTok{for} \NormalTok{idx, prob in }\DataTypeTok{zip}\NormalTok{(}\DataTypeTok{range}\NormalTok{(sample_data.shape[}\DecValTok{0}\NormalTok{]), correctness):}
    \NormalTok{sample_data.ix[idx, }\StringTok{'Correct'}\NormalTok{] = np.random.binomial(n = }\DecValTok{1}\NormalTok{, p = prob)}
\end{Highlighting}
\end{Shaded}

\subsection{Generating reaction times}\label{generating-reaction-times}

Once again, very slow but an extra hurdle to jump through

\begin{Shaded}
\begin{Highlighting}[]
\KeywordTok{for} \NormalTok{idx in }\DataTypeTok{range}\NormalTok{(sample_data.shape[}\DecValTok{0}\NormalTok{]):}
    \NormalTok{sample_data.ix[idx, }\StringTok{'RT'}\NormalTok{] = \textbackslash{}}
      \NormalTok{np.random.lognormal(mean = }\FloatTok{6.0}\NormalTok{, sigma = }\FloatTok{0.5}\NormalTok{) + }\DecValTok{5} \NormalTok{+ \textbackslash{}}
      \NormalTok{np.random.uniform(low = }\DecValTok{10}\NormalTok{, high = }\DecValTok{100}\NormalTok{) + \textbackslash{}}
      \NormalTok{np.random.normal(loc = }\DecValTok{70}\NormalTok{, scale = }\DecValTok{15}\NormalTok{)}
\end{Highlighting}
\end{Shaded}

\subsection{Replacing values selectively with real
data}\label{replacing-values-selectively-with-real-data}

\begin{Shaded}
\begin{Highlighting}[]
\NormalTok{real_data = pd.read_csv(data_dir + data_file)}

\NormalTok{real_rt = real_data.absRT}

\KeywordTok{for} \NormalTok{idx in }\DataTypeTok{range}\NormalTok{(sample_data.shape[}\DecValTok{0}\NormalTok{]):}
    \KeywordTok{if} \NormalTok{sample_data.ix[idx, }\StringTok{'RT'}\NormalTok{] > }\DecValTok{600}\NormalTok{:}
        \NormalTok{sample_data.ix[idx, }\StringTok{'RT'}\NormalTok{] = np.random.choice(real_rt)}
    \KeywordTok{else}\NormalTok{:}
        \KeywordTok{continue}
\end{Highlighting}
\end{Shaded}

\subsection{Creating timed out values}\label{creating-timed-out-values}

\begin{Shaded}
\begin{Highlighting}[]
\CommentTok{# replace values > 2200 with 2200}
\KeywordTok{for} \NormalTok{idx in }\DataTypeTok{range}\NormalTok{(sample_data.shape[}\DecValTok{0}\NormalTok{]):}
    \KeywordTok{if} \NormalTok{sample_data.ix[idx, }\StringTok{'RT'}\NormalTok{] > }\DecValTok{2200}\NormalTok{:}
        \NormalTok{sample_data.ix[idx, }\StringTok{'RT'}\NormalTok{] = }\DecValTok{2200}
    \KeywordTok{else}\NormalTok{:}
        \KeywordTok{continue}
\end{Highlighting}
\end{Shaded}

\subsection{Mapping ages and attempts}\label{mapping-ages-and-attempts}

\begin{Shaded}
\begin{Highlighting}[]

\CommentTok{# map ages}
\NormalTok{sample_data[}\StringTok{'Age'}\NormalTok{] = sample_data.Subject.}\DataTypeTok{map}\NormalTok{(age_dict)}

\KeywordTok{for} \NormalTok{idx in }\DataTypeTok{range}\NormalTok{(sample_data.shape[}\DecValTok{0}\NormalTok{]):}
    \NormalTok{sample_data.ix[idx, }\StringTok{'Attempts'}\NormalTok{] = np.random.random_integers(low = }\DecValTok{1}\NormalTok{, high = }\DecValTok{15}\NormalTok{)}
\end{Highlighting}
\end{Shaded}

\subsection{Asigning to groups}\label{asigning-to-groups}

\begin{Shaded}
\begin{Highlighting}[]

\NormalTok{group_dict = \{\}}
\KeywordTok{for} \NormalTok{subj, idx in }\DataTypeTok{zip}\NormalTok{(subj_ids, }\DataTypeTok{range}\NormalTok{(}\DecValTok{1}\NormalTok{, }\DecValTok{103}\NormalTok{)):}
    \KeywordTok{if} \NormalTok{idx <= }\DecValTok{34}\NormalTok{:}
        \NormalTok{group_dict[subj] = groups[}\DecValTok{0}\NormalTok{]}
    \KeywordTok{elif} \NormalTok{idx > }\DecValTok{34} \NormalTok{and idx <= }\DecValTok{68}\NormalTok{:}
        \NormalTok{group_dict[subj] = groups[}\DecValTok{2}\NormalTok{]}
    \KeywordTok{elif} \NormalTok{idx > }\DecValTok{68}\NormalTok{:}
        \NormalTok{group_dict[subj] = groups[}\DecValTok{1}\NormalTok{]}

\NormalTok{sample_data[}\StringTok{'Group'}\NormalTok{] = sample_data.Subject.}\DataTypeTok{map}\NormalTok{(group_dict)}
\end{Highlighting}
\end{Shaded}

\subsection{Sorting data into long
format}\label{sorting-data-into-long-format}

\begin{Shaded}
\begin{Highlighting}[]

\NormalTok{sample_data = sample_data.sort([}\StringTok{'Subject'}\NormalTok{, }\StringTok{'Age'}\NormalTok{, }\StringTok{'Group'}\NormalTok{, }\StringTok{'Token'}\NormalTok{])}
\end{Highlighting}
\end{Shaded}

\subsection{Changing data to appropriate
types}\label{changing-data-to-appropriate-types}

\begin{Shaded}
\begin{Highlighting}[]
\NormalTok{sample_data.Attempts = np.array(sample_data.Attempts, dtype = }\StringTok{'float64'}\NormalTok{)}
\NormalTok{sample_data.Correct = np.array(sample_data.Correct, dtype = }\StringTok{'float64'}\NormalTok{)}
\NormalTok{sample_data.RT = np.array(sample_data.RT, dtype = }\StringTok{'float64'}\NormalTok{)}
\end{Highlighting}
\end{Shaded}

\subsection{Add in trials}\label{add-in-trials}

\begin{Shaded}
\begin{Highlighting}[]
\NormalTok{trials = }\DataTypeTok{range}\NormalTok{(}\DecValTok{1}\NormalTok{, }\DecValTok{21}\NormalTok{)}
\NormalTok{sample_data[}\StringTok{'Trial'}\NormalTok{] = np.tile(trials, }\DecValTok{40} \NormalTok{* }\DataTypeTok{len}\NormalTok{(subj_ids))}
\end{Highlighting}
\end{Shaded}

\subsection{Generate tokens}\label{generate-tokens}

\begin{Shaded}
\begin{Highlighting}[]
\NormalTok{tokens = }\DataTypeTok{range}\NormalTok{(}\DecValTok{1}\NormalTok{, }\DecValTok{21}\NormalTok{)}
\NormalTok{tokens_repeated = np.repeat(tokens, }\DecValTok{20}\NormalTok{)}
\NormalTok{sample_data[}\StringTok{'Token'}\NormalTok{] = np.tile(tokens_repeated, }\DecValTok{2} \NormalTok{* }\DataTypeTok{len}\NormalTok{(subj_ids))}
\end{Highlighting}
\end{Shaded}

\subsection{Generate phases}\label{generate-phases}

\begin{Shaded}
\begin{Highlighting}[]
\NormalTok{phases = np.repeat([}\DecValTok{1}\NormalTok{, }\DecValTok{2}\NormalTok{], }\DecValTok{400}\NormalTok{)}
\NormalTok{sample_data[}\StringTok{'Phase'}\NormalTok{] = np.tile(phases, }\DecValTok{102}\NormalTok{)}
\end{Highlighting}
\end{Shaded}

\subsection{Export to CSV}\label{export-to-csv}

\begin{Shaded}
\begin{Highlighting}[]
\NormalTok{target_dir = }\StringTok{'/Users/julian/Github/reproducible-research/Day-2/datasets/'}
\NormalTok{filename = }\StringTok{'multivariate-exp-data.csv'}
\NormalTok{sample_data.to_csv(target_dir + filename)}
\end{Highlighting}
\end{Shaded}

\section{Appendix: vectorized
functions}\label{appendix-vectorized-functions}

The data were created in a rather slow, tedious manner, mostly to
introduce control flow and to be explicit about the steps that were
taken. The process can be speeded up greatly however, by taking
advantage of the fact that \texttt{NumPy} and the \texttt{SciPy} stack
are intended to be fast and not very memory-intensive - hence the lack
of deep copies and a lot of modifying in place.

Below are examples of using vectorized functions instead of \texttt{for}
loops to manipulate data.

\subsection{Simulating correctness}\label{simulating-correctness}

\texttt{NumPy} is designed to be fast and (I believe) all of the
function are vectorized. The code to generate correctness, could have
just used the vector of probabilities:

\begin{Shaded}
\begin{Highlighting}[]
\NormalTok{sample_data[}\StringTok{'Correct'}\NormalTok{] = np.random.binomial(n = }\DecValTok{1}\NormalTok{, p = correctness)}
\end{Highlighting}
\end{Shaded}

\subsection{Simulating reaction time}\label{simulating-reaction-time}

For the reaction times, just stipulate the size of the final vector:

\begin{Shaded}
\begin{Highlighting}[]
\NormalTok{sample_data[}\StringTok{'RT'}\NormalTok{] = \textbackslash{}}
  \NormalTok{np.random.lognormal(mean = }\FloatTok{6.0}\NormalTok{, sigma = }\FloatTok{0.5}\NormalTok{, size = }\DecValTok{81600}\NormalTok{) + }\DecValTok{5} \NormalTok{+ \textbackslash{}}
      \NormalTok{np.random.uniform(low = }\DecValTok{10}\NormalTok{, high = }\DecValTok{100}\NormalTok{, size = }\DecValTok{81600}\NormalTok{) + \textbackslash{}}
      \NormalTok{np.random.normal(loc = }\DecValTok{70}\NormalTok{, scale = }\DecValTok{15}\NormalTok{, size = }\DecValTok{81600}\NormalTok{)}
\end{Highlighting}
\end{Shaded}

\subsection{Replacing simulated values with real
values}\label{replacing-simulated-values-with-real-values}

The function \texttt{np.where} could have been used to replace values:

\begin{Shaded}
\begin{Highlighting}[]
\NormalTok{np.where(sample_data.RT > }\DecValTok{600}\NormalTok{, np.random.choice(real_rt), sample_data.RT)}
\end{Highlighting}
\end{Shaded}

\subsection{Creating timed out
values}\label{creating-timed-out-values-1}

In the same vein as the above, create the timed out values

\begin{Shaded}
\begin{Highlighting}[]
\NormalTok{np.where(sample_data.RT > }\DecValTok{2200}\NormalTok{, }\DecValTok{2200}\NormalTok{, sample_data.RT)}
\end{Highlighting}
\end{Shaded}

\end{document}
